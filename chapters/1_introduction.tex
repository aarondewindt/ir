\chapter{Introduction}

% \begin{comment}
% Some information about Dawn
% \end{comment}

The internship assignment consists of researching the process of creating models capable of reproducing the dynamics of the Dawn vehicle during flight. Since it's a fixed wing aircraft the most significant unknown is the aerodynamic model. Digital datcom is used to generate an extensive database of aerodynamic coefficients spanning over a large range of flight conditions. However the coefficients generated by Digital Datcom are an approximation and contain unknown errors which have to be corrected for. 

This report proposes to use real flight data to estimate the parameters of the aerodynamic model using the Two-Step Method. This method splits the non-linear model identification problem into a non-linear flight reconstruction problem and a linear parameter estimation problem. This significantly simplifies the parameter estimation process.

For the correction process the Co-Kriging statistical interpolation technique was chosen. Traditionally used in geostatistics, the CoKriging creates an estimate of a poorly sampled variable with the help of a well sampled, highly correlated secondary variable. In this case the primary variables to be estimated are the real-flight estimated aerodynamic coefficients and the secondary variable are the coefficient from Digital Datcom.

The report starts with some background information about DAWN Aerospace, the company at which the internship is being performed at. It then continuos to explain the work performed during the internship project. First the familiarization tasks, used to get to know the company better and create a detailed definition of the project goals. Then the development of the software necessary to interface with some third party tools from within Python is explained. The report then continuos to explain the system identification method and model mixing method. 
After the work performed is explained, the report summarizes everything in the conclusion and finally a list of recommendations is given on where further work is required.


