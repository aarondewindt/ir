\chapter{Introduction}

\begin{comment}
Some information about Dawn
\end{comment}

The internship assignment consists of researching the process of creating models capable of reproducing the dynamics of the Dawn vehicle during flight. Since it's a fixed wing aircraft the most significant unknown is the aerodynamic model. Digital datcom is used to generate an extensive database of aerodynamic coefficients spanning over a large range of flight conditions. However the coefficients generated by Digital Datcom are an approximation and contain unknown errors which have to be corrected for. 

This report proposes to use real flight data to estimate the parameters of the aerodynamic model using the Two-Step Method (TSM). This method splits the non-linear model identification problem into a non-linear flight reconstruction problem and a linear parameter estimation problem. This significantly simplifies the parameter estimation process.

For the correction process the CoKriging statistical interpolation technique was chosen. Traditionally used in geostatistics, the CoKriging creates an estimate of a poorly sampled variable with the help of a well sampled, highly correlated secondary variable. In this case the primary variables to be estimated are the real-flight estimated aerodynamic coefficients and the secondary variable are the coefficient from Digital Datcom.


